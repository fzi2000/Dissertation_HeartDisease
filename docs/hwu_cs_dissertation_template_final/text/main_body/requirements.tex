\section{Requirement Analysis }
\label{sec:req}

The upcoming section addresses the essential steps needed for the successful implementation of the project. The main functionalities of the proposed system are stated as requirements, which are categorized into Functional and \gls{nfr}. \gls{fr} define what the system should do, while non-functional requirements describe how the system should do it \citep{geeksforgeeks2024functional}. Functional Requirements are further prioritized using the MoSCoW prioritization method and colour-coded for clarity. This analysis delineates the system's primary functions and operational attributes, thereby ensuring a comprehensive understanding of the project's scope. \\

\begin{itemize}
    \item \textbf{M (Must Have):} Necessary features that are critical for the successful completion of the project.
    \item \textbf{S (Should Have):} Important features that should be included if feasible.
    \item \textbf{C (Could Have):} Desirable attributes that are not essential but can be incorporated as enhancements.
    \item \textbf{W (Won’t Have): }Initiatives that are not prioritized within the scope of the project. \\
\end{itemize} 


\subsection{Functional Requirements}
\label{sec:dev_bck}
The functional requirements obtained from the objectives of this study are mentioned below. 
% Define colors
\definecolor{mred}{RGB}{255, 102, 102}   % Light red for M
\definecolor{cgreen}{RGB}{102, 255, 102} % Light green for C
\definecolor{sorange}{RGB}{255, 178, 102} % Light orange for S
\begin{longtable}{|p{2.5cm}|p{9cm}|p{2cm}|}
\hline
\textbf{FR} & \textbf{Description} & \textbf{Priority} \\
\hline
FR-1 & \textbf{Feature Selection: }
Perform feature engineering to identify the most significant factors that influence cardiovascular disease outcomes. & \cellcolor{mred}\textbf{M} 
\hline
FR-2 & \textbf{Machine Learning Model Development: }
Develop a predictive model using machine learning algorithms, such as Logistic Regression and Random Forests, specifically for cardiovascular disease prediction. & \cellcolor{mred}\textbf{M} 
\hline
FR-3 & \textbf{Risk Stratification Implementation: }
Integrate risk stratification techniques within the model to categorize patients into high-risk, medium-risk, or low-risk groups. & \cellcolor{mred}\textbf{M}  \hline
FR-4 & \textbf{Model Performance Evaluation: }
Evaluate the model’s performance using various metrics such as accuracy, precision, recall, F1 score,\gls{auc}, and confusion matrix. & \cellcolor{mred}\textbf{M}  \hline \hline
FR-5 & \textbf{Explainable AI Integration: }
Incorporate explainable AI techniques, such as SHAP or LIME to assess the model's interpretability. & \cellcolor{mred}\textbf{M} 
\hline
FR-6 &\textbf{ Risk Assessment Visualization Generation: }
Create visualizations, such as graphs, to illustrate how each feature contributes to the risk predictions for individual patients. & \cellcolor{sorange}\textbf{S} 
\hline
FR-7 & \textbf{Prototype Development: }
Develop a graphical user interface that enables healthcare professionals to input patient data and get risk predictions in a user-friendly manner. & \cellcolor{cgreen}\textbf{C} 
\hline
FR-8 & \textbf{User Interface Functionality: }
The system could enable healthcare professionals to navigate through results using a graphical user interface. & \cellcolor{cgreen}\textbf{C} 
\hline
\caption{Functional Requirements}
\end{longtable}

\subsection{Non-Functional Requirements}
Non-Functional Requirements list the quality attributes of the system.
\vspace{-0.25cm}
\begin{longtable}{|p{2.5cm}|p{9cm}|p{2cm}|}
\hline
\textbf{FR} & \textbf{Description} & \textbf{Priority} \\
\hline
NFR-1 & \textbf{GDPR-Compliant Security: }
The dataset used in this research is open source and complies with GDPR regulations governing data protection and ethical standards. & \cellcolor{mred}\textbf{M} 
\hline
NFR-2 & \textbf{Code Documentation: }
The code written will be documented to allow easy readability and maintainability.  & \cellcolor{mred}\textbf{M}
\hline
NFR-3 & \textbf{Performance Efficiency: }
The model will be optimized for performance to ensure efficient data processing. & \cellcolor{sorange}\textbf{S} 
\hline
NFR-4 & \textbf{Response Time: }
The system shall process user inputs and deliver predictions within 10 seconds to provide an effective user experience. & \cellcolor{sorange}\textbf{S} 
\hline
\caption{Non-Functional Requirements}
\end{longtable}
\vspace{1cm}
\subsection{Traceability Matrix}
A traceability matrix provides a framework to map each research objective with the corresponding functional requirements. This approach ensures a clear connection between the objectives of the study and the specific system functionalities.
\begin{table}[h!]
\centering
\begin{tabular}{|p{4.5cm}|c|c|c|c|c|c|}
\hline
\textbf{Requirement/Objectives} & \textbf{O1} & \textbf{O2} & \textbf{O3} & \textbf{O4} & \textbf{O5} & \textbf{O6} \\
\hline
FR-1 & \checkmark &   &  & \checkmark  &  &  \\
\hline
FR-2 &  &  & \checkmark    &  & \checkmark &  \\
\hline
FR-3 &  &  &  &  & \checkmark &  \\
\hline
FR-4 &  &  &  &  &  & \checkmark \\
\hline
FR-5 &  &  &  & \checkmark & \checkmark & \checkmark \\
\hline
FR-6 &  &  &  &  & \checkmark & \\
\hline
FR-7 &  &  &  &  & \checkmark &  \\
\hline
FR-8 &  &  &  &  &  & \checkmark \\
\hline
NFR-1 & \checkmark & \checkmark &  &  &  &  \\
\hline
NFR-2 &  &  & \checkmark  &  & \checkmark &  \\
\hline
NFR-3 &  &  &  &  &  & \checkmark \\
\hline
NFR-4 &  &  &  &  &  & \checkmark \\
\hline
\end{tabular}
\caption{Traceability Matrix}
\label{tab:traceability_matrix}
\end{table}

\subsection{Research Questions } 
\label{subsec: research qs}
The main research questions discussed in this report are listed below. \\
% \vspace{-0.2 cm}
\begin{enumerate}
    \item What are the key risk factors associated with cardiovascular diseases, and how do they influence disease prediction outcomes?
    \item How do various machine learning models and deep learning models compare in their effectiveness for predicting cardiovascular disease?
    \item Which predictive features significantly contribute to the accuracy of cardiovascular disease predictions, and how can these features be effectively identified and validated?
    \item How can the incorporation of explainable AI techniques, such as \gls{shap} or \gls{lime}, enhance the interpretability of the model’s predictions?
\end{enumerate}