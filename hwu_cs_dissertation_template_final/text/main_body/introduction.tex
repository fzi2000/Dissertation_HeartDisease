\section{Introduction}
\label{sec:intro}

This research explores \gls{cvd} prediction and patient risk assessment using \gls{ml} techniques.

\subsection{Motivation}
\label{sec:intro_motiv}

The heart is considered as one of the most vital organs in the human body. It is responsible for circulating blood throughout the body. According to the World Health Organization, cardiovascular diseases are one of the main causes of death worldwide. It results in about 17.9 million deaths every year \citep{who2019cardiovascular}. In the United States alone, an individual experiences a heart attack every 40 seconds \citep{cdc2024heart}. Predictive models for early Cardiovascular Disease identification have become crucial because of the rising growth of risk factors such as diabetes, obesity and sedentary activities. \\
\noindent This project has been driven by the potential of harnessing machine learning techniques to improve the early diagnosis of cardiovascular diseases. Predictive models would help healthcare professionals identify people at a high risk of heart disease and provide personalised treatment plans. This strategy can greatly minimize the financial burden on healthcare systems and improve patient outcomes. Additionally, this project incorporates \gls{xai} to make the model’s predictions more understandable, building physicians’ confidence in applying machine learning tools in clinical settings.


\subsection{Aim and Objectives}
\label{sec:intro_aim}
\textbf{Aim } \\
The aim of this project is to enhance existing efforts in cardiovascular disease prediction by developing an explainable machine learning model. This project seeks to provide healthcare professionals with a better understanding of the predictions by incorporating Explainable AI and risk stratification. It also aims to evaluate the performance of traditional machine learning and deep learning algorithms. Hence, this project aims to make a significant contribution to the domain of cardiovascular health. \\

\noindent \textbf  {Objectives } \\
The key objectives are listed below. \\

\noindent \textbf{O1: Conduct a detailed study on cardiovascular diseases} - Research and understand the basic concepts of \gls{cvd} and its risk factors.

\noindent \textbf{O2: Perform \gls{eda}}- Apply appropriate data preprocessing techniques to ensure the dataset used in this study is of high quality.

\noindent \textbf{O3: Evaluate Established Machine Learning Models-} Investigate and evaluate the suitability of various models, such as Logistic Regression, Random Forests, and Deep learning techniques, for predicting cardiovascular disease. 

\noindent \textbf{O4: Optimal Feature Selection-} Identify significant predictive features whose presence improves the accuracy of \gls{cvd} prediction.

\noindent \textbf{O5: Machine Learning Model Development- }Develop a model for predicting cardiovascular diseases with various \gls{ml} algorithms, deep learning techniques, and risk stratification.

\noindent \textbf{O6: Model Performance Evaluation-} Evaluate the developed model to check its predictive accuracy and interpretability by employing relevant evaluation metrics.


\subsection{Contributions}
\label{sec:intro_contrib}

This project contributes to the following fields in disease detection in healthcare. 
\begin{enumerate}
    \item Model development for cardiovascular disease prediction.
    \item Integration of Explainable AI techniques.
    \item Risk stratification for patients.
\end{enumerate}

\subsection{Organisation}
\label{sec:intro_orga}
This report is structured to progressively illustrate the flow of the research. \\ \vspace{0.3 cm}
\textbf{Introduction: }The introduction gives an overview of the project with its aims and objectives. 
\textbf{Background:} This section presents the fundamental concepts of the project in sub-sections, followed by a detailed literature review of prior work in the field. It ends with a critical analysis of the existing literature. \\
\textbf{Requirements: }The Requirements Analysis chapter lists the study's objectives through functional requirements and outlines the non-technical aspects via non-functional requirements. This includes a traceability matrix that links these requirements to the project's objectives. \\
\textbf{Design: The Research Methodology} section offers an overview of how the research will be executed in the upcoming semester. The \textbf{Evaluation Strategy} section details the metrics and measures that will be employed to assess the model’s performance. \\
\textbf{References:} A list of references used in this study is included at the end. \\
\textbf{Appendices:} The appendices include the \gls{ples} and Project  Management sections. Appendix A presents the steps taken to address \gls{ples} and Appendix B outlines the risks and the project plan for implementation. Finally, Appendix C comprises project journals from weeks 4,7 and 10 which document incremental progress made during the study. 