\section{Appendix: Professional, Legal, Ethical and
Social Issues (PLES)}
\label{app:uml}
The use of machine learning for cardiovascular disease prediction raises many professional, legal, ethical, and social considerations. Taking these aspects into account is crucial to ensure that the research adheres to the industry standards, legal practices and ethical concerns. This project will adhere to data protection regulations, uphold ethical standards in model development, and consider the broader societal impact to promote responsible AI in cardiovascular disease prediction. \\
\subsection{Professional}
The project adheres to the British Computer Society's code of conduct, ensuring integrity and professionalism. It provides code readability and proper maintenance of code. This study aligns with best practices in software development and data science, adhering to high standards in programming and research methodology. All code written will be documented. The data used is sourced from reputable and open-source databases. No personal or sensitive information will be collected during the course of this project. Open-source Python libraries like Scikit-learn, Matplotlib, and TensorFlow will be used. The project has a proper structure, adequate time for tasks, and regular meetings with the supervisor. Project risks have been considered, and appropriate mitigation strategies have been developed. Secondary data is licensed for research purposes, and ideas and information are cited and referenced. The use of any content from both online and offline sources will be cited and referenced. The project's scope is confined to developing a Machine Learning model for academic and exploratory purposes, eliminating client or customer involvement and additional professional issues.

\\
\subsection{Legal}
All information used in this study including the dataset, source code, ideas and any work that is not my own has been given due credit by citing and referencing. All research papers or articles are either open-source or have been granted permission for usage. These papers have been referenced in the bibliography. The project will use open-source tools and platforms for its development.

\subsection{Ethical}
This project aims to develop a machine-learning model that predicts the presence of cardiovascular disease. The dataset contains human information, but the dataset is open-source and protected by open-source license rules. It has been obtained from reputable sources and has been cited in previous academic articles. Human information is anonymized and unlinked to ensure the privacy and security of people. An ethics form stating the research aims and methods has been approved by Heriot-Watt University. This ensures that the research aligns with the Data Protection Act (DPA) and ethical guidelines. As the study focuses on system performance metrics and doesn’t involve human subjects, ethical data collection tools are not required.

\subsection{Social}
This system is used for Cardiovascular disease prediction. Hence, its social impact is large. This project is developed primarily for research and academic purposes and is not intended for immediate public use. If the project is used, all results must be consulted with a healthcare professional and must not be the sole decision-making factor. Source code access will be limited to the supervisor and the research evaluators at Heriot-Watt University who will grade the project. Hence, concerns related to misuse of the application or authorized use of source code are not applicable.